\chapter{DEFINING SYSTEM-LEVEL PROPERTIES}
\thispagestyle{plain}

\label{Defining}

% Definition of a system-level property
System-level properties play a central role in this dissertation.
A system-level property is a mathematical measurement of some non-explicit concept of interest in a agent-based model.
These properties are typically non-explicit, meaning inferring their values from just the given agent-level configuration parameters is difficult.

% Discuss how system-level properties are identified
% Discuss how system-level properties are mathematically defined
Identifying which system-level properties of an ABM to analyze is the first step of using the \framework.
Deciding on which features to be measured is a job for the user of the framework and will be influenced in what that user finds interesting or what that user needs to know.
Each system-level parameter must be defined as a mathematical measurement of the behavior in terms of observable features of the ABM.
For example, the average number of sheep in the Wolf Sheep Predation model is a system-level property of the system that is calculated by averaging the number of sheep measured in each time step over the lifetime of the system.
The framework utilizes the user provided mathematical definition to create the data set that is used in the forward mapping.

% Give an outline of the chapter
In this chapter, I discuss the problem of defining system-level properties for use in \fw.
First, the important ``stability assumption" and the consequences of using assumption is explained.
Next, comprehensive list of common classes of system-level properties are outlined.
This list should be useful to users of \fw in defining their own system-level properties.
Then, the process in which \fw uses to sample an ABM is explained.
Finally, software implementation details are given for this particular step of \fw.


\section{The Stability Assumption}

% Before delving into the topic of defining system-level properties, I first discuss this assumption
Before delving into the topic of defining system-level properties, I first discuss the ``stability assumption."
Understanding this assumption and its consequences is important in defining system-level properties so that they will be measured accurately by \fw.

% A measurement of a system-level property should be stable.
% Explain what stable means
%  the sampling error is normally distributed 
% For example, the percentage of trees burned down
\fw assumes that the naturally occurring error for sampled values of a system-level property must have the following feature: the expected mean and expected median of the behaviors should be identical.
In other words, the errors should be expected to be evenly distributed around some value $\bar y$, in  both magnitude and quantity.
Errors following a normal distribution will have this property.
For example, the percentage of trees burned down in the Fires\footnote{More on the Fires model is discussed in Section \ref{sec:Fires}} model will vary from run to run, but for the most part will be centered around one value for a particular configuration.

% Explain the need for stability - we are trying to develop a functional mapping to solve the forward-mapping problem. Without this assumption, we cannot use regression(see background).
This assumption is important because if this assumption does not hold, regression algorithms will converge to predict with bias.
As more points are sampled, the regression algorithms will converge on the value provided by the mean of measured values.
However, the median may be a better predictor of the expected behavior if the errors are not normally distributed.
This problem is best explained with an example.

% This is an example explaining the need for stability
For example, in the Wolf Sheep Predation domain, one simple system-level measurement could be the average number of wolves after a thousand time steps.
This number can be misleading, because certain configurations will \textit{sometimes} result in the wolves going extinct (i.e., zero wolves).
Other times, the same configurations will result in the wolves converging to a stable non-zero population.
When the wolf population does stabilize, it converges upon generally the same population value, $\hat w$.
The average of successive runs will result in a value between zero and this stable population size.
Unfortunately, this average is not useful in any way, since it does not tell how often the wolves will go extinct or what their population will be, should it be the case that the population stabilizes.
This fact is illustrated in Figure X.

%% Have a figure here that shows a histogram of wolf populations given the same configuration.
%% It will show a large number of wolf populations at zero and then a bell-curve like structure near the true value.
%% Show the median and the mean

To remedy this problem, the average can be decomposed into two components: the case of going extinct and the case of not.
The average number of wolves $\bar w$ would be:
\[\mathrm{E}(\bar w) = \mathrm{P}(extinct) * (\bar w | extinct) + (1 - \mathrm{P}(extinct)) * (\bar w | \neg extinct) \]
where $extinct$ is whether the wolves went extinct or not, $\mathrm{P}(extinct)$ is the probability wolves go extinct, and $(\bar w | \neg extinct)$ is the average number of wolves, given they did not go extinct.
The value of $(\bar w | extinct)$ is zero, so the left hand side of the above equation is zero.
Therefore, the equation can be simplified to:
\[\mathrm{E}(\bar w) = (1 - \mathrm{P}(extinct)) * (\bar w | \neg extinct) \]

As stated earlier, $\bar w$ is not very stable because sometimes it is zero and sometimes it is $(\bar w | \neg extinct)$.
However, the values $\mathrm{P}(extinct)$ and $(\bar  w | \neg extinct)$ are useful in describing the system's behavior and are more stable (and accurate) than just using $\bar w$.
In conclusion, the system-level properties that should be measured for the Wolf Sheep Predation model is not the average number of wolves, but the average number of wolves given that they did not go extinct, and the probability of the wolves going extinct.
To calculate the value of $\mathrm{P}(extinct)$, divide the number of samples that exhibited zero wolves by the number of total samples for that configuration.
To calculate the value of $(\bar w | \neg extinct)$, average the population size of the wolves, only in samples in which they did not go extinct.

% In most situations, a nonstable measurement could be the result of two situations: there is an underlying threshold effect or biased readings before convergence.
In most situations, non-stable system-level properties are the result of some sort of \textit{threshold effect}.
Threshold effects are sudden changes in behavior, given a small change in the configuration.
Threshold effects are also sometimes referred to as \textit{tipping points}.
In ABMs that behave randomly, configurations that lay on a boundary of a threshold effect can exhibit erratic behavior, which was the case with the Wolf Sheep Predation example.
Typically, this problem can be overcome by decomposing the property into several sub-properties that describe the threshold effect. 
This approach is what was needed to provide useful and accurate information in the Wolf Sheep Predation example.


\section{Common Classes of System-Level Properties}

% Many system-level properties can fit into a certain category

\subsection{Average of a Value}
% average value over the life of a system

\subsection{Variance of a Value}
% variance of a value over the life of a system

\subsection{Probability of an Event}
% probability of an event occurring in the system


\subsection{Measuring a Value That Changes Over Time}
% There are a number of properties that change over time that may want to be measured
% So far, all properties have had scalar values,
% therefore, it may seem that it is impossible to measure behaviors over time, since these properties change.
% With a change of perspective, this is possible.

% to measure a system-level behavior that changes over time, the system-level properties used by \fw are scalars that represent this behavior in a parametric model.
% The parameters are the system-level properties are learned in the forward mapping process and can be used to predict the parameters of the model.

% For example, consider a NetLogo Traffic Simple simulation, in which cars move in a linear space; traffic jams appear in waves. We would like to develop a mapping for the red car.
% from visual observations, it appears that the red car has a minimum speed and a maximum speed.
% We know that the acceleration is linear from the programming of the model.
% Also, can see that it follows a rhythmic behavior: the car reaches a certain speed, then stops.
% With these, we have all that we need to create a wave-like parametric model that describes this behavior.
% The following variables will be used in the parametric wave:
%   Let $A$ be the amplitude of the wave (i.e., the maximum velocity minus the minimum velocity)
%   Let $T$ be the period of the wave (i.e., the time between a max velocity and a min velocity)
%   Let $h$ be the height of the wave (i.e., the minimum speed)
% The behavior inside each period is linear, so it can be described as a simple linear model:
%   \[\displaystyle \frac{A}{T} t + h\]
% Since this behavior is periodic, the modulus can be taken of the time, (period):
%   \[\displaystyle \frac{A}{T} (t mod T) + h \]


\section{Sampling}
% Two major steps:
%  1. Retrieve raw data from the simulation
%  2. Compute on that data to generate the properties of interest

% \fw typically uses a evenly distributed random sampling technique.
%  this is useful for changing the size of the dataset dynamically
% More advanced sampling techniques could be used in the future.

% Sometimes... Discuss the need for doing several runs on the same configuration

% Discuss the selection of number of steps the ABM should run



\section{Implementation Details}
% Steps:
%  1. identify what properties are needed in the simulation and the ranges of values we want to configure --- that are needed to perform the system-level properties
%  2. Sample the ABM to generate a raw data set
%  3. Pass the raw data set through a script that converts the raw entries into the desired system-level properties to create a new data set

% First step: the sampling program is written in java
%  the sampling program is given the ranges that are to be sampled,
%  and the data values that are needed

% The second step: the sampling program performs the sampling
%  the raw data is returned to standard out. The data is tab delimited to separate items within entries. Entries are delimited by newlines.
%  The output file could be redirected to a file to be stored for future use, if necessary


% The third step: use the ``map" script to parse the raw data, and output a data set that contains the system-level properties.
%  the map script parses the data for each entry.
%    the parse returned is a hash table object where the keys are the names of the raw data items and the values are the values inside, as strings.
%  the map script is passed a user-created python module that is used to compute the values of the system-level properties, from the raw data.
%  this user-created python module has a list of functions that each returns a value for a system-level property.
%  map returns a new data set to standard out, containing the configuration parameters and the system-level properties.
%  simple example
%  list example
%  More details on how to create this python module is covered in Appendix XX.


 







